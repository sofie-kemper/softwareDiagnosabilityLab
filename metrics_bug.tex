\subsection{Bug Characteristics}

The following metrics are related to a known bug and aim to quantify features of
the bug such as its location and size. The following  statistics are taken from
the Defects4J Dissection (see \ref{sec:defects4jDissection}).

In addition, for all static metrics analysed via TeamScale on the project level,
we define buggy-file versions (prefix BF-) which measure the corresponding
statistic only on the files that are responsible for the bug, i.e., those files
that are modified in the bug's corresponding patch. The aggregation of these
metrics for multiple files is achieved in the same way as the project-level
aggregation, i.e., some findings are summed up whereas others are normalised by
the respective source lines of code or the maximum values are used (e.g.,
maximum cyclomatic complexity number).

On top of that, the dynamic metrics we defined on the project level are also
extended to the bug level (prefix B-) by looking at bug-specific properties of
the call- and data-dependency graphs, e.g., the in-degree of the vertices which
correspond to the faulty methods. In cases where there are multiple faulty
methods, the \enquote{most extreme} value is used, e.g., the maximum in-degree
or the minimum eccentricity.

Other ideas such as analysing whether the bug is part of a design antipattern or
calculating object-oriented metrics on the bug method were rejected due to their
high effort and low presumed meaningfulness for the software diafnosability.

\subsubsection{Number of Files (B-NF)}

This metric quantifies the number of files modified in the patch that fixes the
bug.

\subsubsection{Number of Classes (B-NC)}

The number of classes changed in the patch is given by this metric.

\subsubsection{Number of Methods (B-NM)}

This metric indicates the number of methods modified in the patch.

\subsubsection{Number of Lines (B-NL)}

The metric counts the total number of lines modified in the patch.

\subsubsection{Number of Lines Added/Removed/Modified (B-NAL/B-NRL/B-NML)}

These metrics indicate the number of lines (of code) that were added, removed,
or modified, respectively, in the patch. The sum of these three metrics
corresponds to the number of lines (B-NL).

\subsubsection{Number of Chunks (B-NCH)}

This metric indicates the number of \enquote{chunks} modified in the patch. A
chunk is defined as a section in the code containing sequential line changes.

\subsubsection{Number of Repair Patterns (B-NRP)}

This metric quantifies the number of different \enquote{repair patterns}
employed in the patch for the corresponding bug. A repair pattern is a general
pattern of fixing a bug, i.e., a characteristic of the bug patch. Examples
include  \enquote{Single Line}, indicating that only a single line was changed,
or \enquote{Wrong Variable Reference}.

\subsubsection{Number of Repair Actions (B-NRA)}

The number of repair actions counts the number of different actions employed in
the process of fixing the bug, i.e., visible in the patch diff. Examples of
repair actions include \enquote{Variable replacement by another variable} or
\enquote{Conditional (if) branch addition}.
