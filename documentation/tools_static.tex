\subsection{Lizard}
\label{sec:lizard}

Lizard is a Python-based tool that analyses code size and perceived code
complexity as well as parameter and token counts at the function level. It can
analyse Java, C/C++, JavaScript, Python, Ruby, Swift, PHP, Scala, and Objective
C scripts.  More information can be found at
\url{https://pypi.org/project/lizard/}.

\subsection{TeamScale}
\label{sec:teamscale}

TeamScale is a tool that provides a multitude of static code analyses to
indicate code quality and possible quality defects (\enquote{findings}). It is
free for academic usage. The tool monitors the quality of code over time by
using manualyy checked-in versions or automatically detected versions of a given
repository. By using the diffs between versions, it can quickly analyse the
history of a project. The tool FindBugs (see
\url{http://findbugs.sourceforge.net}) is also contained in TeamScale. The
metrics are calculated on the file or even method level and aggregated
hierarchically. Hence, we can obtain project metrics easily.

More information regarding TeamScale can be found at
\url{https://www.cqse.eu/en/products/teamscale/landing/}.

We have implemented a REST-client to automatically query the metric values from
TeamScale and to transform them into a format we can easily use for our
analysis.

\subsection{JPeek}
\label{sec:jpeek}

JPeek is a static collector of Java code metrics relating to cohesion and
coupling. It measures 5 metrics on the method level and aggregates this data
quantitatively, e.g., calculating min, max, variance, and other statistical
measures,  as well as qualitatively, i.e., \enquote{scoring} components as a
whole and categorising them into the three classes \enquote{green},
\enquote{yellow}, and \enquote{red}. In addition, it defines and measures
\enquote{defects} which are classes whose scores particularly bad using the mean
scores as baseline.  

We use the version JPeek 0.26.3. For more information on the tool, the
interested reader is referred to \url{http://www.jpeek.org} and
\url{https://github.com/yegor256/jpeek}.
