\subsection{Dynamic Metrics}

The following metrics are \emph{dynamic} design metrics, i.e., they are
calculated based on a program's call or data dependency graph as generated by
the tool JDCallgraph (see \ref{sec:jdcallgraph}). All metrics are computed on
the data dependency graph (prefix DD-) as well as the callgraph (prefix CG-)
using the statistical programming language R and the igraph library (see
\ref{sec:igraph}). We use the multi-edge versions of these graphs and simplify
them where appropriate (see below).

\subsubsection{Vertex Count (DD-VC, CG-VC)}

The vertex count is a measure of graph size. It quantifies the number of
vertices in the graph.

\subsubsection{Edge Count (DD-EC, CG-EC)}

The edge count is a measure of graph size, indicating the number of edges
(including multi-edges) between pairs of vertices of the graph.

\subsubsection{Simplified Edge Count (DD-SEC, CG-SEC)}

The simplified edge count corresponds to the number of edges in the graph, not
counting multi-edges, i.e., if there are multiple edges between one pair of
vertices, only one of these edges is counted.

\subsubsection{Multi-Edge Proportion (DD-MEP, CG-MEP)}

This metric indicates the proportion of multi-edges in the graph, i.e., which
fraction of the edges present in the graph are actually multi-edges.

\subsubsection{Maximum Vertex Degree (DD-MAXVD, CG-MAXVD)}

The maximum vertex degree is the maximum total degree of any vertex in the
graph.

\subsubsection{Mean Vertex Degree (DD-MVD, CG-MVD)}

The mean vertex degree is the average of all vertex total degrees in the graph.

\subsubsection{Vertex Degree Quantiles (DD-VD*Q, CG-VD*Q)}

We look at the vertex degree distribution by aggregating the data in statistical
quantiles. The 90-th, 80-th, 75-th and 50-th (median) quantiles are used. For
instance, the 90-th quantile (denoted by CG-VD90Q) indicates the threshold
below which 90\% of all vertex degrees lie.

\subsubsection{Maximum Vertex In-Degree (DD-MAXVI, CG-MAXVI)}

The maximum vertex in-degree is the maximum in-degree of any vertex in the
graph, where in-degrees are the number of directed edges with the given vertex
as end point.

\subsubsection{Mean Vertex In-Degree (DD-MVI, CG-MVI)}

This metric is calculated as the average of all vertices' in-degrees.

\subsubsection{Maximum Vertex Out-Degree (DD-MAXVO, CG-MAXVO)}

The maximum vertex out-degree indicates the maximum number of outgoing edges for
all vertices in the graph.

\subsubsection{Mean Vertex Out-Degree (DD-MVO, CG-MVO)}

This metric denotes the average of all vertices' out-degrees.

\subsubsection{Mean Start-Node Degree (DD-MSND, CG-MSND)}

This metric denotes the average degree of the start nodes. Start nodes are the
test cases, i.e., those nodes in the graph which possess only outgoing edges. It
indicates how many methods are called directly by the test cases.

\subsubsection{Graph Diameter (DD-GD, CG-GD)}

The graph diameter is a measure of a graph's size, indicating the geatest
distance between any pair of vertices.

\subsubsection{Graph Radius (DD-GR, CG-GR)}

The graph radius denotes the minimum eccentricity of any vertex in the graph.
The eccentricity of a vertex indicates how far this vertex is from that vertex
most distant from it in the graph.

\subsubsection{Mean Distance (DD-MD, CG-MD)}

This metric indicates the mean (geodesic) distance between all pairs of nodes in
the graph.

\subsubsection{Maximum Eigenvector Centrality (DD-MEC, CG-MEC)}

A vertex' eigenvector centrality indicates its importance or influence in the
graph, e.g., nodes that are connected to many nodes, especially to many other
important nodes, show a high eigencentrality. Each vertex has a relative score.
Google's PageRank is a randomised version of eigenvector centrality which is
more robust. 

This metric denotes the maximum eigencentrality found in the graph.

\subsubsection{Eigenvector Centrality Quantiles (DD-EC*Q, CG-EC*Q)}

This metric characterises the distribution of vertices' eigenvector centrality
scores by providing the 90-th, 80-th, 75-th, and 50-th quantiles (see above) of
all the eigencentrality scores in the graph.

\subsubsection{Vertex Connectivity (DD-VCON, CG-VCON)}

The vertex connectivity is a metric that indicates how many vertices need to be
removed to render the graph disconnected, i.e., ensure that every vertex can no
longer reach every other vertex. It might indicate how dense the graph, and
thus, how complex its call- or data-dependency-structure, is.

\subsubsection{Edge Connectivity (DD-ECON, CG-ECON)}

A graph's edge connectivity indicates how many edges need to be removed from the
graph to create a disconnected graph.

\subsubsection{(Average) Clustering Coefficient (DD-CC, CG-CC)}

A vertex' clustering coefficient captures the local connectivity, i.e., it
measure how likely it is that any two of its neighbour's are connected amongst
each other. This score is aggregated via averaging over all vertices in the
graph to get a clustering coefficient for the entire graph, which indicates the
extent to which clusters exist in the graph.